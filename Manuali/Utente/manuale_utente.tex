\documentclass[10pt, a4paper]{report}

% --- PREAMBOLO ---
\usepackage[utf8]{inputenc}
\usepackage[T1]{fontenc}
\usepackage[italian]{babel}
\usepackage{graphicx} 
\graphicspath{{images/}} % Decommenta se le immagini sono in una sottocartella
\usepackage{float}    
\usepackage{geometry} 
\usepackage{xcolor}   
\usepackage{listings} 
\usepackage{hyperref} 

% Margini
\geometry{top=1.5cm, bottom=1.5cm, left=1.5cm, right=1.5cm}

% Setup Link
\hypersetup{
	colorlinks=true,
	linkcolor=black,
	filecolor=magenta,      
	urlcolor=blue,
	pdftitle={Manuale Utente - BookRecommender},
}

% Comandi specifici per il contenuto del Manuale Utente
\newcommand{\gui}[1]{\textbf{\textit{#1}}} 
\newcommand{\filename}[1]{\texttt{#1}}     

% Intestazione

\title{\textbf{Documentazione di Progetto}\\Manuale Utente}
\author{
	\textbf{Book Recommender} \\ \\
	Lorenzo Monachino - 757393 - VA \\
	Lyan Curcio - 757579 - VA \\
	Sergio Saldarriaga - 757394 - VA\\
	Nash Guizzardi - 756941 - VA
}
\date{Anno Accademico 2024/2025}

\begin{document}
	
	% Generazione Titolo
	\maketitle
	
	% -------------------------------------------------------------------
	% INDICE
	% -------------------------------------------------------------------
	\tableofcontents
	\newpage
	
	% -------------------------------------------------------------------
	% CAPITOLO 1: INTRODUZIONE
	% -------------------------------------------------------------------
	\chapter{Introduzione}
	
	\section{Scopo del documento}
	Questo documento rappresenta il \textbf{Manuale Utente} per l'applicazione \textit{BookRecommender}. Il manuale è strutturato per guidare l'utente dall'installazione dell'ambiente fino all'utilizzo delle funzionalità avanzate, distinguendo tra utenti non registrati e registrati.
	
	\section{Descrizione del prodotto}
	``Book Recommender'' è una soluzione software implementabile in svariati contesti che permette sia agli amministratori che agli utenti di avere un rapporto diretto con i libri disponibili nel database.
	
	L'interfaccia grafica è stata progettata con uno stile moderno e minimale (palette colori: \textit{Dark Blue} e \textit{White}), garantendo un'esperienza utente intuitiva grazie a una barra laterale di navigazione sempre presente.
	
	% -------------------------------------------------------------------
	% CAPITOLO 2: INSTALLAZIONE
	% -------------------------------------------------------------------
	\chapter{Installazione}
	
	\section{Requisiti di sistema}
	Per garantire il corretto funzionamento di \textit{BookRecommender}, è necessario soddisfare i seguenti requisiti.
	
	\subsection{Sistema Operativo}
	L'applicazione è stata progettata e testata sui seguenti sistemi operativi:
	\begin{itemize}
		\item Windows 10 / 11
		\item ArchLinux (HyperLand 0.53)
		\item Linux Fedora 43 (KDE Plasma 6.5.4)
		\item ArchLinux (KDE Plasma 6.5.5)
	\end{itemize}
	\textit{Nota: Il funzionamento su piattaforme differenti (es. macOS o altre versioni Linux) non è garantito.}
	
	\subsection{Software richiesto}
	\begin{itemize}
		\item \textbf{Java:} È essenziale avere installato correttamente \textbf{Java 25}. Versioni precedenti non sono state testate.
		\item \textbf{PostgreSQL:} È necessario aver installato \textbf{PostgreSQL 18} per il funzionamento del database (gestito tramite gli script di inizializzazione).
		\item \textbf{Connessione di rete:} Necessaria per la comunicazione Client-Server.
	\end{itemize}
	
	\section{Setup ambiente e Installazione}
	L'applicazione non richiede un wizard di installazione classico. Il software è distribuito tramite archivi compressi contenenti gli eseguibili.
	
\begin{enumerate}
	\item Scaricare e decomprimere la cartella del progetto in una posizione locale (es. \texttt{Documenti/BookRecommender}).
	\item Assicurarsi che all'interno siano presenti i file di avvio (\texttt{.bat} per Windows o \texttt{.sh} per Linux/Mac).
	\item Creare un nuovo database su PostgreSQL denominato \texttt{bookrecommender} eseguendo il comando SQL: \texttt{CREATE DATABASE bookrecommender;}
\end{enumerate}
	
	% -------------------------------------------------------------------
	% CAPITOLO 3: ESECUZIONE ED USO
	% -------------------------------------------------------------------
	\chapter{Esecuzione ed Uso}
	
	\section{Setup e lancio del programma}
	L'applicazione segue un'architettura Client-Server. La procedura di avvio deve seguire un ordine preciso per garantire che i componenti comunichino correttamente.
	
	\subsection{Avvio su Windows}
	\begin{enumerate}
		\item Aprire la cartella contenente il software.
		\item \textbf{Primo avvio (Inizializzazione Database):} Fare doppio clic sul file \filename{InitDB.bat}. Questo passaggio va eseguito \textbf{una sola volta} per creare e popolare il database.
		\item \textbf{Avvio Server:} Fare doppio clic su \filename{Server.bat}. Attendere che il server sia attivo.
		\item \textbf{Avvio Client:} Dopo aver avviato il server, fare doppio clic su \filename{Client.bat}.
	\end{enumerate}
	\textit{Nota: Per chiudere il server, digitare \texttt{Ctrl+C} nel terminale e confermare con ``S''.}
	
	\subsection{Avvio su Linux / macOS}
	Si consiglia di eseguire i comandi da terminale per visualizzare eventuali errori grafici o di permessi.
Eseguire i seguenti comandi da terminale:
\begin{lstlisting}[language=bash, frame=single, numbers=none]
	$ cd path/to/BookRecommender
	$ ./InitDB.sh    # Solo al primo avvio (DB Setup)
	$ ./Server.sh    # Avvia il Server (non chiudere)
	$ ./Client.sh    # Avvia il Client (in un nuovo terminale)
\end{lstlisting}
	
	All'avvio, si presenterà la schermata principale (Home).
	
	\begin{figure}[H]
		\centering
		\includegraphics[width=0.8\textwidth, keepaspectratio]{ben_no_log.png}
		\caption{Schermata iniziale ``Catalogo Libri''.}
	\end{figure}
	
	% -------------------------------------------------------------------
	% CAPITOLO 4: USO DELLE FUNZIONALITÀ (Tutti gli utenti)
	% -------------------------------------------------------------------
	\chapter{Uso delle funzionalità (Utenti non registrati)}
	Dalla schermata iniziale, chiunque può ricercare libri nel catalogo e visualizzarne i dettagli.
	
	\section{Ricerca Libri}
	La schermata \textbf{Catalogo Libri} presenta una barra di ricerca avanzata.
	\begin{enumerate}
		\item \textbf{Selezione Filtro:} Utilizzare il menu a tendina per scegliere il criterio:
		\begin{itemize}
			\item \gui{Titolo}: Abilita solo il campo di testo principale.
			\item \gui{Autore}: Cerca tutti i libri di uno specifico autore.
			\item \gui{Autore e Anno}: Abilita sia il campo di ricerca principale (per l'autore) che il campo secondario \gui{Anno}.
		\end{itemize}
		\item \textbf{Inserimento:} Digitare i termini di ricerca.
		\item \textbf{Avvio:} Cliccare sul pulsante \gui{Cerca}.
	\end{enumerate}
	I risultati appariranno nella lista centrale.
	
	
	\section{Visualizzazione Dettagli Libro}
	Per visualizzare informazioni approfondite:
	\begin{enumerate}
		\item Selezionare un libro dalla lista dei risultati.
		\item Premere il pulsante \gui{Info Libro} situato in basso a destra.
	\end{enumerate}
	
	La schermata \textbf{Informazioni Libro} è suddivisa in aree tematiche:
	\begin{itemize}
		\item \textbf{Intestazione:} Titolo, Autore e Anno di pubblicazione.
		\item \textbf{Colonna Sinistra (Recensioni):} Mostra la lista delle recensioni lasciate dagli utenti e un pannello riepilogativo con le \textit{Statistiche Voti}.
		\item \textbf{Colonna Destra (Consigli):}
		\begin{itemize}
			\item \textit{Consigliati dagli utenti:} Lista dei libri suggeriti spontaneamente dalla community.
			\item \textit{Top Consigliati:} Una classifica dei suggerimenti più popolari.
		\end{itemize}
	\end{itemize}
	
	\begin{figure}[H]
		\centering
		\includegraphics[width=0.8\textwidth, keepaspectratio]{info_lib.png}
		\caption{Schermata dettagli libro con statistiche e consigli.}
	\end{figure}
	% -------------------------------------------------------------------
	% CAPITOLO 5: USO DELLE FUNZIONALITÀ (Utenti registrati)
	% -------------------------------------------------------------------
	\chapter{Uso delle funzionalità (Utenti registrati)}
	Per accedere alle funzioni avanzate (creazione librerie, recensioni, consigli), è necessario autenticarsi.
	
	\section{Registrazione}
	\begin{enumerate}
		\item Dalla barra laterale, cliccare su \gui{Registrati}.
		\item Compilare il modulo \textbf{Crea il tuo Account} con:
		\begin{itemize}
			\item \gui{Nome} e \gui{Cognome}
			\item \gui{UserID} (identificativo univoco)
			\item \gui{Codice Fiscale}
			\item \gui{Email}
			\item \gui{Password} e \gui{Conferma Password}
		\end{itemize}
		\item Cliccare su \gui{Registrati}. Il sistema notificherà eventuali errori di validazione nei campi.
	\end{enumerate}
	
	\begin{figure}[H]
		\centering
		\includegraphics[width=0.8\textwidth, keepaspectratio]{reg.png}
		\caption{Modulo di registrazione utente.}
	\end{figure}

	
	
	\section{Login e Area Utente}
	Cliccando su \gui{Accedi} e inserendo \gui{UserID} e \gui{Password}, si effettua il login.
	La barra laterale cambierà stato, mostrando il nome dell'utente e sbloccando i pulsanti:
	\begin{itemize}
		\item \gui{Crea Libreria} (Accesso rapido)
		\item \gui{Le tue Librerie}
		\item \gui{Logout}
	\end{itemize}
	
	\begin{figure}[H]
		\centering
		\includegraphics[width=0.8\textwidth, keepaspectratio]{log.png}
		\caption{Schermata login.}
	\end{figure}
	
	\section{Gestione Librerie Personali}
	
	\subsection{Creazione di una libreria}
	Dalla sezione \gui{Le tue Librerie}:
	\begin{enumerate}
		\item Localizzare il box \textbf{Crea Nuova Libreria} in alto.
		\item Inserire il nome desiderato nel campo di testo.
		\item Cliccare su \gui{Crea}. La nuova libreria apparirà immediatamente nell'\textit{Elenco Librerie Esistenti}.
	\end{enumerate}
	
	\begin{figure}[H]
		\centering
		\includegraphics[width=0.8\textwidth, keepaspectratio]{all_my_libs.png}
		\caption{Schermata creazione libreria.}
	\end{figure}
	
	\subsection{Aggiungere un libro a una libreria}
	\begin{enumerate}
		\item Cercare un libro e aprire la scheda \gui{Info Libro}.
		\item Cliccare sul pulsante espandibile \gui{Aggiungi a una Libreria} situato in basso.
		\item Selezionare la libreria di destinazione dal menu a tendina.
		\item Cliccare su \gui{Conferma Aggiunta}.
	\end{enumerate}
	
			\begin{figure}[H]
		\centering
		\includegraphics[width=0.8\textwidth, keepaspectratio]{save_libr.png}
		\caption{Schermata salvataggio libro in libreria.}
	\end{figure}
	
	\subsection{Gestione interna della Libreria}
	Selezionando una libreria dall'elenco e cliccando \gui{Apri Libreria}, si accede alla vista dettagliata.
	Qui è possibile:
	\begin{itemize}
		\item Vedere la lista dei libri salvati.
		\item \textbf{Rimuovere un libro:} Selezionarlo e premere \gui{Rimuovi}.
		\item \textbf{Consigliare:} Selezionare un libro e premere \gui{Consiglia} per suggerirlo ad altri utenti.
		\item \textbf{Visualizzare i propri dati:} La barra laterale destra mostra automaticamente:
		\begin{itemize}
			\item \textit{I tuoi Consigliati:} Libri che hai suggerito correlati al libro selezionato.
			\item \textit{La tua Recensione:} Il testo della tua valutazione, se presente.
		\end{itemize}
	\end{itemize}
	
		\begin{figure}[H]
	\centering
	\includegraphics[width=0.8\textwidth, keepaspectratio]{sugg.png}
	\caption{Schermata consiglia.}
\end{figure}

		\begin{figure}[H]
	\centering
	\includegraphics[width=0.8\textwidth, keepaspectratio]{gest_lib.png}
	\caption{Schermata gestione librerie.}
\end{figure}
	
	\section{Valutazione dei Libri}
	Un utente registrato può recensire i libri presenti nelle proprie librerie.
	\begin{enumerate}
		\item All'interno della propria libreria, selezionare un libro.
		\item Cliccare su \gui{Valuta}.
		\item Compilare la scheda \textbf{Scrivi la tua Recensione} assegnando un voto da 1 a 5 per i seguenti parametri:
		\begin{itemize}
			\item \textbf{Stile}
			\item \textbf{Contenuto}
			\item \textbf{Gradevolezza}
			\item \textbf{Originalità}
			\item \textbf{Edizione}
		\end{itemize}
		\item Per ogni parametro è possibile aggiungere una nota testuale opzionale.
		\item È presente anche un campo per \textbf{Note Finali} complessive.
		\item Cliccare su \gui{Conferma Recensione} per salvare.
	\end{enumerate}
	
		\begin{figure}[H]
	\centering
	\includegraphics[width=0.8\textwidth, keepaspectratio]{val.png}
	\caption{Schermata valutazione.}
\end{figure}
	
	% -------------------------------------------------------------------
	% CAPITOLO 6: RISOLUZIONE PROBLEMI (NUOVO per principianti)
	% -------------------------------------------------------------------
	\chapter{Risoluzione Problemi Comuni}
	In caso di difficoltà durante l'uso dell'applicazione, consultare questa sezione per le soluzioni ai problemi più frequenti.
	
	\section{Il programma non si avvia}
	\begin{itemize}
		\item \textbf{Causa:} Java non è installato correttamente o la versione è obsoleta.
		\item \textbf{Soluzione:} Verificare di aver installato \textbf{Java 25} aprendo il terminale e digitando \texttt{java -version}.
	\end{itemize}
	
	\section{Impossibile connettersi al Server}
	\begin{itemize}
		\item \textbf{Sintomo:} Il Client mostra messaggi di "Connessione rifiutata" all'avvio.
		\item \textbf{Soluzione:} Assicurarsi di aver avviato il file \filename{Server.bat} (o \filename{.sh}) \textit{prima} di avviare il Client. Controllare che la finestra del Server sia aperta e non mostri errori.
	\end{itemize}
	
	\section{Nessun libro trovato nella ricerca}
	\begin{itemize}
		\item \textbf{Causa:} Errore di connessione al database.
		\item \textbf{Soluzione:} Assicurarsi di aver inizializzato il database tramite il rispettivo eseguibile.
	\end{itemize}
	
	% -------------------------------------------------------------------
	% CAPITOLO 7: CONCLUSIONI
	% -------------------------------------------------------------------
	\chapter{Data set, Limiti e Conclusioni}
	
	\section{Data set di test}
	Al fine di facilitare la valutazione del software e l'esperienza utente immediata, il sistema include un data set di test predefinito.
	L'esecuzione dello script di inizializzazione (\filename{InitDB}) popola il database con una selezione di libri, autori e utenti di prova. Questo permette di effettuare ricerche e visualizzare dettagli senza la necessità di inserire manualmente un catalogo iniziale.
	
	\section{Limiti della soluzione sviluppata}
	Come indicato nella sezione di installazione, l'applicazione è garantita solo sui sistemi operativi specificamente testati (Windows 10/11, ArchLinux, Fedora).
	L'uso su macOS o altre distribuzioni Linux potrebbe comportare instabilità, errori grafici o problemi nell'apertura dei file, dovuti alla gestione delle finestre o dei percorsi di sistema (file system).
	
	\section{Conclusioni}
	Il sistema ``Book Recommender'' si presenta come una soluzione efficace per la gestione e la valorizzazione dell'esperienza di lettura. Unendo funzionalità di ricerca, catalogazione e recensione dettagliata, favorisce il coinvolgimento degli utenti e offre alle aziende uno strumento per monitorare il gradimento delle opere letterarie.
	
	\section{Contatti}
	Per supporto tecnico o segnalazioni, fare riferimento al team di sviluppo:
	\begin{itemize}
		\item \textbf{Nash Guizzardi} (Project Manager): \href{mailto:nguizzardi@studenti.uninsubria.it}{nguizzardi@studenti.uninsubria.it}
		\item \textbf{Lyan Curcio}: \href{mailto:lcurcio@studenti.uninsubria.it}{lcurcio@studenti.uninsubria.it}
		\item \textbf{Sergio Saldarriaga}: \href{mailto:ssaldarriaga@studenti.uninsubria.it}{ssaldarriaga@studenti.uninsubria.it}
		\item \textbf{Lorenzo Monachino}: \href{mailto:lmonachino@studenti.uninsubria.it}{lmonachino@studenti.uninsubria.it}
	\end{itemize}
	
	% -------------------------------------------------------------------
	% BIBLIOGRAFIA (Aggiornata stile Manuale Tecnico)
	% -------------------------------------------------------------------
	\begin{thebibliography}{9}
		
		\bibitem{java}
		Oracle,
		\textit{Java Documentation},
		\url{https://docs.oracle.com/en/java/}
		
		\bibitem{postgresql}
		PostgreSQL Global Development Group,
		\textit{PostgreSQL Documentation},
		\url{https://www.postgresql.org/docs/}
		
		\bibitem{latex}
		The LaTeX Project,
		\textit{LaTeX Project Documentation},
		\url{https://www.latex-project.org/}
		
	\end{thebibliography}
	
\end{document}